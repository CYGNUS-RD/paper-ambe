%%%%%%%%%%%%%%%%%%%%%%%%%%%%%%%%%%%%%%%%%%%%%%%%%%%%%%%%%%%%%%%%%%%%%%%%
%    INSTITUTE OF PHYSICS PUBLISHING                                   %
%                                                                      %
%   `Preparing an article for publication in an Institute of Physics   %
%    Publishing journal using LaTeX'                                   %
%                                                                      %
%    LaTeX source code `ioplau2e.tex' used to generate `author         %
%    guidelines', the documentation explaining and demonstrating use   %
%    of the Institute of Physics Publishing LaTeX preprint files       %
%    `iopart.cls, iopart12.clo and iopart10.clo'.                      %
%                                                                      %
%    `ioplau2e.tex' itself uses LaTeX with `iopart.cls'                %
%                                                                      %
%%%%%%%%%%%%%%%%%%%%%%%%%%%%%%%%%%
%
%
% First we have a character check
%
% ! exclamation mark    " double quote  
% # hash                ` opening quote (grave)
% & ampersand           ' closing quote (acute)
% $ dollar              % percent       
% ( open parenthesis    ) close paren.  
% - hyphen              = equals sign
% | vertical bar        ~ tilde         
% @ at sign             _ underscore
% { open curly brace    } close curly   
% [ open square         ] close square bracket
% + plus sign           ; semi-colon    
% * asterisk            : colon
% < open angle bracket  > close angle   
% , comma               . full stop
% ? question mark       / forward slash 
% \ backslash           ^ circumflex
%
% ABCDEFGHIJKLMNOPQRSTUVWXYZ 
% abcdefghijklmnopqrstuvwxyz 
% 1234567890
%
%%%%%%%%%%%%%%%%%%%%%%%%%%%%%%%%%%%%%%%%%%%%%%%%%%%%%%%%%%%%%%%%%%%
%
\documentclass[12pt]{iopart}
\newcommand{\gguide}{{\it Preparing graphics for IOP Publishing journals}}
%Uncomment next line if AMS fonts required
%\usepackage{iopams}  
\usepackage{graphicx} 
\usepackage{xcolor}
\usepackage{lineno}
\usepackage{cite}
\usepackage{xspace}
\linenumbers

\begin{document}


\title[]{Identification of  nuclear recoils in gas  with a sCMOS camera}


\newcommand{\fe}{\ensuremath{^{55}\textrm{Fe}}\xspace}
\newcommand{\ambe}{\ensuremath{^{241}\textrm{Am} - ^9\textrm{Be}}\xspace}
\newcommand{\lemon}{\textsc{LEMOn}\xspace}

\author{A Marco Emanuele, Cavoto Gianluca, Pinci Davide}

\address{San Miguel, Mexico}
\ead{emanuele.a.marco@roma1.infn.it}
\vspace{10pt}
\begin{indented}
\item[]May 2020
\end{indented}

\begin{abstract}

\end{abstract}

%
% Uncomment for keywords
%\vspace{2pc}
%\noindent{\it Keywords}: XXXXXX, YYYYYYYY, ZZZZZZZZZ
%
% Uncomment for Submitted to journal title message
%\submitto{\JPA}
%
% Uncomment if a separate title page is required
%\maketitle
% 
% For two-column output uncomment the next line and choose [10pt] rather than [12pt] in the \documentclass declaration
%\ioptwocol
%



\section{Introduction}

The advent of a market of high position resolution and single photon  light sensors can open new opportunity to investigate ultra-low rate phenomena as Dark Matter  (DM) particle  scattering on nuclei in a gaseous  target.

The nature of DM is still one of the key  issues to understand  our Universe. Different models  predicts the existence of neutral particles with a mass of GeV  or higher that would fill our Galaxy. They  could interact with the nuclei present in ordinary matter producing highly ionizing nuclear recoils but with a  kinetic energy as small as  few keV. Moreover, given the motion of the Sun in the Milky Way towards the Cygnus constellation such nuclear recoils would exhibit a dipole angular distribution in a terrestrial detector.
In this paper we describe the use of a scientific CMOS camera to capture the light emitted by Gas Electron Multipliers (GEMs) in a Time Projection Chamber (TPC) device. The GEMs are located in the TPC gas volume at the anode position and are used to convert the ionization produced in the gas by   the  nuclear recoils into flashes of visible light. The flash of light can be located in space and its shape adopting a cluster  recognition algorithm. Neutron $\gamma$ radiation emitted by radioactive source are used to  set in motion  atomic electrons and nuclei respectively in the gas volume. Moreover, natural radiation as cosmic rays is leaving a trail of ionization in the gas. They are all producing different  patterns of light emission from the GEMs that can be reconstructed and analyzed. Nuclear recoils can then be efficiently identified down to few keV kinetic energy. 
 The study of the optical readout of TPC has been recently conducted with several small size prototypes (NITEC~\cite{JINST:nitec}, ORANGE~\cite{NIM:Marafinietal, bib:jinst_orange2}, \lemon~\cite{bib:eps, bib:ieee17, bib:elba}) with various particles sources. In the following, we report the study of nuclear recoils excited by neutron from a  Am-Be source and electron recoils from a $^{55}Fe$ source in the gas volume of the  \lemon prototype.

 \section{Experimental layout and data }
 A 7 liter active drift volume TPC  (named \lemon  \cite{paperBTF} ) was employed to detect the particles recoils. It 
features   a  200$\times$240~mm$^2$ elliptical field cage with a 200 mm distance between the anode and the cathode. The anode side is instrumented with a 200$\times$240~mm$^2$ rectangular triple  GEM structure.
Standard LHCb-like \cite{bib:thesis} GEMs  (70~$\mu$m diameter holes and 140~$\mu$m pitch) were used with two 2~mm wide transfer gaps between them. The light emitted from the GEMs is detected with   an ORCA-Flash 4.0 camera \cite{ORCAcamera} through a  $203\times254\times1$ mm$^3$ transparent window and a  bellow with a tunable length (Fig.\ref{fig:LemonShielded}.  This camera is positioned  at a 52 cm  distance from the outemost  GEM layer and is based on a sCMOS sensor with a high granularity ($2048\times2048$ pixels), very low noise (around two photons per pixel), high sensitivity (70\%  quantum efficiency at  600~nm) and good linearity. This camera is instrumented with a Schneider lens (with an aperture f/0.95 and a focal length of 25~mm). The lens is placed at a distance $d$ of 50.6 cm from the last GEM
in order to obtain a de-magnification
$\delta = (d/f) - 1 = 19.25$ to
image a surface $25.6 \times 25.6$~cm$^2$ onto the
$1.33 \times 1.33$~cm$^2$ sensor.
In this configuration, each pixel
 is therefore imaging  an effective area of 125$\times$125~$\mu$m$^2$ of the GEM layer. The fraction of the light collected by the lens can be evaluated \cite{bib:jinst_orange1} to be $1.7 \times 10^{-4}$.

A semi-transparent mesh was used as a cathode in order to collect light on that side also with a 50$\times$50~mm$^2$ HZC Photonics XP3392 photomultiplier \cite{PMTPhotonics} (PMT) detecting light through a transparent $50\times50\times4$~mm$^3$ fused silica window. More details can be found in ...


 
\begin{figure}[ht]
	\centering
	\includegraphics[width=0.45\linewidth]{LEMON-Shielded.jpg}
  	\caption{\lemon with the lead shield of the  drift volume cage. The sCMOS camera (on the front) is looking at the GEMs through a blackened bellow.}
  	\label{fig:LemonShielded}
\end{figure}



Typical images frame obtained with a 30 ms exposure time of the sCMOS
camera are shown in Fig.~\ref{fig:typicalimage1}. Several light spots
are visible due to different ionization particles interacting in the
gas.  In these examples, Fig.~\ref{fig:typicalimage1} (left) shows an
image with typical long tracks from cosmic rays travelling through the
full gas volume, where clusters of higher energy deposition are
clearly visible, superimposed to likely natural radiactivity events.
Fig.~\ref{fig:typicalimage1} (right) shows an example of a clean event
with one straight cosmic ray track, that can be used for energy
calibration purposes.

The events shown in Fig.~\ref{fig:signals} show images recorded with
the same 30 ms exposure time, but in presence of radioactive
sources. Figure~\ref{fig:signals} (left) shows one example of several
energy spots, characteristic of 5.9 keV energy deposits of a \fe
radioactive source.  This is the standard candle for calibration and
performance evaluation of the detector, and its extensive usage for
\lemon characterization is documented in
Ref.~\cite{bib:fe55}. Figure~\ref{fig:signals} (right) shows an event
recorded in presence of the \ambe radioactive source, which produces
$\alpha$-particles and neutrons, which are candidates serving as
generators of nuclear recoils.
 
\begin{figure}[ht]
  \begin{center}
    \includegraphics[width=0.49\linewidth]{figures/pic_run02317_ev8_oriIma_paper}
    \includegraphics[width=0.49\linewidth]{figures/pic_run02156_ev527_oriIma_paper}
    \caption{Two typical pictures taken with the sCMOS camera with a 30
      ms exposure time. Left: cosmic tracks and natural radioactivity
      signals are present. Right: two long cosmic rays tracks are
      present, observed in a run without any source.
      \label{fig:typicalimage1}}
  \end{center}
\end{figure}

 
\begin{figure}[ht]
  \begin{center}
    \includegraphics[width=0.49\linewidth]{figures/pic_run01843_ev93_oriIma_paper}
    \includegraphics[width=0.49\linewidth]{figures/pic_run02317_ev342_oriIma_paper}
    \caption{Two pictures taken with the sCMOS camera with a 30 ms
      exposure time. Left: picture taken in presence of \fe radioactive
      source. Right: a nuclear recoil candidate is present, in an image
      with \ambe radioactive source, together with signals from natural
      radioactivity.      \label{fig:signals}}
  \end{center}
\end{figure}

\lemon was operated in an overground location at Laboratori Nazionali di Frascati (LNF) with  a He-CF$_4$ (60/40) gas mixture, the triple GEM system set at a voltage across the GEM sides of xxx~V and an electric field between them of x.0 kV/cm - using a HV GEM power supply \cite{Corradi:2007df} ensuring stability and accurate monitoring of the bias currents. The gas mixture was kept at atmospheric pressure under continuous flow of about xxx cc/min and with the GEMs operated at $ x\times10^5$ gain. The typical photon yield for this  type of gas mixtures has been measured to be around  0.07 photons per avalanche electron.\cite{bib:jinst_orange1, bib:roby, bib:tesinatalia}

The field cage was powered by a CAEN N1570.\cite{CAENN1570} generating an electric field  of 0.6 kV/cm. 


\textcolor{red}{The ORCA Camera I/O has been configured in order to get a pre-trigger, that must occur 80~$\mu$s before the shutter, and to synchronize the PMT signal waveform acquired with an oscilloscope LeCroy 610Zi. Optics and exposure time (30~ms) were optimized to ensure the largest light collection and to avoid events due to the natural radioactivity. Between 100 and 300 images were typically acquired per run. }

A  5cm thick lead shielding was mounted around the \lemon field cage to reduce the natural radioactivity background. From the measurements of the GEM current with and without the lead shielding a reduction of the total ionization in the sensitive gap, very likely due to external radioactivity of a factor 2 was estimated.

 An neutron source, based on a 3.5~$\times~$10$^3$~MBq activity  $^{241}$Am source contained in a Beryllium capsule ({\it AmBe}) was placed at a distance of 50~cm from the sensitive volume side. 
 Because of the interactions between $alpha$ particles produced by the $^{241}$Am and the Beryllium nucluei, {\it AmBe} source isotropically emits:
 \begin{itemize}
     \item photons with an energy of 59~keV produced by $^{241}$Am;
     \item neutrons with a kinetic energy mainly in a range between 1 to 10 MeV;
     \item photons with an energy of 4~MeV produced along with neutrons in the interaction between $\alpha$s and Be.
 \end{itemize}
 
The presence of lead shield around the sensitive volume absorbed almost completely the 59~keV photon component. A small faction of them reached the gas trough small slits accidentally present between the bricks.
 

\clearpage
 
\section{Cluster pattern recognition}
\label{sec:clustering}
The light produced in the multiplication process in the GEM and
collected by the CMOS sensor is gathered in clusters of neighboring
pixels, following the energy deposition of the particle travelling
through the gas of the detector. The energy of the particle
originating the deposit is estimated by the amount of the light
collected by the sensor.  Therefore, it is of primary importance to
have a reconstruction algorithm that fully collects the pixels
impinged by true photons originating from the energy deposits, while
rejecting most of the electronics noise. This can either create fake
clusters or, more likely, add pixels in the periphery of clusters from
real photons, biasing the energy estimate.

The energy reconstruction follows a three-steps procedure: the
single-pixel noise suppression is briefly described in
Section~\ref{sec:zerosuppression}. This is followed by the proper or
real clustering: first the basic clusters of single small deposits is
described in Section~\ref{sec:basiccl}, then the superclusters, aiming
to follow the track patterns for depositions extended in space, and
seeded by the basic clusters, is described in
Section~\ref{sec:supercl}.

The results of this papers are based on the properties of the
reconstructed superclusters, described in
Section~\ref{sec:clustershapes}.


\subsection{Noise suppression}
\label{sec:zerosuppression}
The electronic noise of the sensor was estimated in data-taking runs acquired with
the sensor in complete dark ({\it pedestal} runs). For each pixel, the
pedestal was computed as the average of the counts over many frames,
while the electronic noise was estimated as their standard deviation
(SD). The distribution of the pixels SD is shown in
Fig.~\ref{fig:noise}. The mode of this distribution is about 1.8
photons per pixels, but a tail is present, with pixels having a noise
of more than 5 photons per pixels. For such pixels, a very
non-Gaussian distribution was observed, while for the pixels in the
bulk of the distribution, the pedestal distribution followed a Gaussian
shape. To form the pedestal-subtracted image, the pedestal mean
$\mu_i$ was subtracted to the image for each $i^{th}$ pixel.  An initial
noise suppression was applied by neglecting the pixels with counts less
than $1.3\times\textrm{SD}_i$.
%where $i$ represents the
%pixel (\textit{zero suppression}). 
%
\begin{figure}[ht]
  \centering
  \includegraphics[width=0.45\linewidth]{figures/sensor_noise}
  \caption{Distribution of the electronic noise of the sensor,
    estimated in images taken with sensor in complete dark, and
    evaluated as the SD of the distribution of the counts for each
    pixel.  \label{fig:noise}}
\end{figure}
%
On such pedestal-subtracted zero-suppressed images an upper threshold
was applied to reject hot pixels, which are more likely due to sensor
instabilities than to energy deposition. These were found  to be not
malfunctioning pixels since they disappeared after a power cycle of the
camera: therefore a dynamic (run-by-run) suppression was needed.
They were efficiently identified as high-intensity, isolated pixels,
and distinguished by a true energy deposit, for which each pixel is
surrounded by some other active pixels. A threshold was applied on the
ratio $R_9$ between the pixel and the average of the counts in a
$3\times3$ pixels matrix surrounding it, and a minimum number of two
pixels above noise in that matrix was required to discriminate  good from
hot pixels.

The resolution of the resulting image was then reduced by forming
\textit{macro-pixels}, by averaging the counts in $4\times4$ pixel
matrices. This was needed to reduce the combinatorics   of the subsequent
clustering algorithm to be executed   in a reasonable time for each
image. On such $512\times512$ pixel map, a median
filtering~\cite{medianfilter} was applied, as described in more details
in Ref.~\cite{medianfilter_cygno}. The output image is passed to the
basic clustering algorithm, described in the following.





\subsection{Basic clusters reconstruction}
\label{sec:basiccl}
The basic clustering algorithm is described in full details in
Ref.~\cite{iDBSCAN}, and it is briefly summarized here.
The energy deposition pattern in the active volume of the TPC,
and thus its two-dimensional (2D) projection on the $x--y$ axes 



\subsection{Super clusters reconstruction}
\label{sec:supercl}
The aim of the superclustering procedure is to collect the majority of
the pixels belonging to a track which is long and, eventually, with
an irregular pattern. The main limitation of \idbscan to follow a long
track is mainly originated by the non uniform energy deposition along
the path length.  As can be clearly seen in
Fig.~\ref{fig:basic_clusters} (right), or even in the example of a raw
image of an event with two long cosmic rays in
Fig.~\ref{fig:typicalimage1} (right), clusters with larger energy
release are followed by regions along the path with a lower or even a
zero release.  These local minima are sometimes as large, in the 2D
space, as the typical size of the $\epsilon$ parameter of
\dbscan~\cite{dbscan}. Despite the low electronic noise of the
ORCA-Flash 4.0 camera sensor, the energy releases in these local
minima are similar in magnitude to the average single-pixel noise.
The \idbscan is limited in connecting the full length of an extended
path, because of two reasons. First, inflating $\epsilon$ parameter as
much as needed to cover the areas of local minima conflicts with the
need to reject noise around the cluster.  The \idbscan parameters were
optimized for the \lemon running conditions to collect most of the
signals of $E \approx 5$\keV and to reject the typical noise of
$\approx 1$ photon per pixel. This avoids collecting extra noise in
the cluster, biasing the energy scale and worsening its resolution,
and keeps the rate of fake clusters at a negligible level.  This is
studied in great detail in Ref.~\cite{iDBSCAN}.  Second, the iterative
nature of the algorithm with very different parameters for each
iteration, each tuned for very different intensity, makes it
convenient and efficient for a deposition of a fixed energy density
(like the spots originating from the \fe source), but not for the
cases as in Fig.~\ref{fig:basic_clusters} (right), where the same
track is split in several parts, with some of them in different
iterations.  This requires a method that can continuously follow the
pattern of the track, profiting of the full resolution image, where
the {\it gradients} of the energy deposition along the track
trajectory are smaller than the ones in the transverse
direction. Moreover, executing any of the most common clustering methods on the
full $1024\times1024$ image is not manageable CPU-wise, due to the huge
pixel combinatorics.

The procedure adopted for the final supercluster reconstruction in the
\lemon detector started from defining the \textit{interesting regions}
in the image that may contain pixels from an energy deposit. These are
identified by the basic cluster algorithm \idbscan previously
described, which is applied on the $512\times512$ reduced-resolution
image. In order to gather the peripheral pixels, especially along the
track trajectory where breaks into small basic clusters may have happened,
a window of $5\times5$ pixels is considered, around each pixel
belonging to a macro-pixel clustered in a basic cluster. A full
resolution image formed only by the interesting pixels passing the
simple initial filtering described in Sec.~\ref{sec:zerosuppression}
was created.  The gradients of the intensity $N_{ph}$ in such image were
computed pixel-by-pixel to look for the edge region where the image
turns from signal to noise-only:
%
\begin{equation}
\label{eq:gradient}
\vert\vert\nabla(N_{ph})\vert\vert =
\sqrt{\left(\frac{\partial N_{ph}}{\partial x}\right)^2
  +\left(\frac{\partial N_{ph}}{\partial y}\right)^2},
\end{equation}
%
while the gradient direction is given by:
\begin{equation}
  \label{eq:graddir}
  \theta = \tan^{-1}\left(\frac{\partial N_{ph}}{\partial y}/\frac{\partial N_{ph}}{\partial x}\right).
\end{equation}
%
In order to reduce the effect of the noise which makes the first
derivatives in Eq.~\ref{eq:gradient} to fluctuate, a Gaussian filter
is applied, with a 5$\sigma$ threshold, where $\sigma$ is the SD of
the intensities of the pixels considered.

The superclustering algorithm, applied on the filtered image, is an
application of the \textit{morphological geodesic active
contours}\cite{gac,mgac}, called \gac in the following.  This method
uses an active contour finding, widely used in computer vision, where
the boundary curve $\mathcal{C}$ of an object is detected by
minimizing the \textit{energy} $E$  associated to $\mathcal{C}$:
\begin{equation}
  \label{eq:gacenergy}
  E(\mathcal{C}) = \int_{0}^{1} g(N_{ph})(\mathcal{C}(p)) \cdot \vert\mathcal{C}_p\vert dp,
\end{equation}
%where $N_{ph}$ is the number of photons in the pixel,
where $ds=\vert\mathcal{C}_p\vert dp$ is the arc-length parameterization of
the curve in the 2D space, and $g$ is the stopping edge function,
which allows to select the boundary of the cluster.  In the \gac
method used for the \lemon images, the $g$ function is purely
geometrical, and uses the geodesics of the image, \ie, the local
minimal distance path between points with the same gradient, defined
before. The function $g(N_{ph})$ is given by:
\begin{equation}
g(N_{ph}) = \frac{1}{\sqrt{1+\alpha\vert\nabla G_\sigma * N_{ph}\vert}},
\end{equation}
which is minimal in the edges of the image.  The $G_\sigma * N_{ph}$ is the
aforementioned $5\sigma$ Gaussian filter, and the parameter $\alpha$,
which regulates the strength of the filtering was tuned on
typical \lemon images to be $\alpha=100$.

This method was chosen because it allows to follow track patterns
that may vary from convex to concave shape, eventually with kinks, \eg
in cases of $\delta$-ray emissions. To improve the shrinking of the
cluster boundary in the cases of tracks turning from concave to convex
along their trail, the \textit{balloon} force~\cite{mgac} is set to
-1, in order to push the contour towards a border in the areas where
the gradient is too small. A number of 300 iterations is used to
evolve the supercluster contour.

The example track shown in Fig.~\ref{fig:basic_clusters} (right) after
the basic clustering step, is shown again in full resolution, zoomed
around the cluster, in Fig.~\ref{fig:super_clusters1} (left). The
output of the superclustering with the \gac algorithm is shown on the
right panel of the same figure. The splitting of the clustering,
present after the basic cluster step, was recovered. The portions with
high density and low density along the path of the energy deposition
were joined together. Other three examples of superclustered images are
shown in Fig.~\ref{fig:super_clusters2}, in runs without any artificial
radioactive source. The top left panel shows an example of a cosmic
ray track fully reconstructed by the \gac superclustering, which also
includes a $\delta$-ray in the middle of the track length. The top
right panel shows an example of curly track from a candidate of
natural radioactivity interaction; bottom panel shows an example where
both a cosmic ray and a curly track are present. In this case, the
extremes of the long and straight track are still split, but this is
much rarer than after the basic clustering, and it happens when the
local minimal along the trajectories are compatible with noise-only
for more than $\approx$1\unit{cm}.
%
\begin{figure}[ht]
  \begin{center}
     \includegraphics[width=0.49\linewidth]{figures/pic_run02317_ev8_oriIma_paper_zoom}
      \includegraphics[width=0.49\linewidth]{figures/pic_run02317_ev8_sc_3D_paper}
      \caption{Left: zoom on the full-resolution image of a
        track candidate in a run with the \ambe radioactive
        source. Right: output of the superclustering on the rebinned
        image. \label{fig:super_clusters1}}
  \end{center}
\end{figure}
%
\begin{figure}[ht]
  \begin{center}
     \includegraphics[width=0.49\linewidth]{figures/pic_run02156_ev49_sc_3D_paper}
     \includegraphics[width=0.49\linewidth]{figures/pic_run02156_ev641_sc_3D_paper} \\
     \includegraphics[width=0.6\linewidth]{figures/pic_run02156_ev631_sc_3D_paper}
     \caption{Superclusters reconstructed in a run without artificial radioactive
       sources.  Top left: cosmic ray track fully reconstructed by the \gac
       superclustering. A $\delta$-ray is included in the supercluster. Top right: curly track from a candidate of
       natural radioactivity interaction. Bottom: a cosmic ray with the
       extremes not joined to the main track, plus a curly track from
       natural radioactivity. \label{fig:super_clusters2}}
  \end{center}
\end{figure}



\section{Cluster observables}
\label{sec:clustershapes}
 
 Define  (projected) length $l$, light $L$, energy $E$ (light after calibration), density of light $\delta$ (and then $\frac{dE}{dl}$ after calibration but projected that's why I am using $dl$ !), slimness $S$
 
 Describe here briefly the $E$ calibration ? saturation effect removal ? 
 
\section{Nuclear recoil identification results}
 
 During the data-taking xxx frames were recorded in absence of any external source ({\it no-source} sample). In these frames the interaction of ultra-relativistic cosmic ray particles (mostly muons) are clearly visible  as  very long cluster. Internal radioactivity of the \lemon materials  also contribute several smaller size cluster.  It is then possible to define a pure sample of cosmic ray tracks by requiring $L$ > 13 cm and $S$ < 0.1  \textcolor{red}{che altro manca? }. The cosmic ray cluster identified with this selection show small values of  $\delta$ $\sim $ 5 well compatible with the small specific ionization of ultra-relativistic particles. 
 In Fig.\ref{fig:cosmics} we show the distribution of the observed $\frac{dE}{dl}$ for the no-source sample and for the Am-Be samples. The broadening of the distribution is mainly due to the specific energy loss fluctuation in the gas mixture of the cosmic ray particles.   Its mean value corrected for the effect of the angular distribution of the cosmic rays is   xx keV/cm that is in good agreement with the Garfield prediction of 2.3 keV/cm.  The angle with the horizontal axis is evaluated by  measuring the distance between \textcolor{red}{EAM spiega come hai fatto}. In Fig.\ref{fig:cosmics} the distribution   of the  angle of the cosmic ray cluster in the no-source sample is shown.
 
 
 \begin{figure}[ht]
	\centering
	\includegraphics[width=0.45\linewidth]{dEdx_cosmics.png}
	\includegraphics[width=0.45\linewidth]{cosmic_angle.png}
  	\caption{Left: Distributions of reconstructed energy release per centimetre for cosmic rays. Right: Distributions of reconstructed angles for cosmic rays. }
  	\label{fig:cosmics}
\end{figure}
\textcolor{red}{Non farei vedere il Ferro in questa figura. Inoltre il plot delle differenza deve essere fatto su scala diversa. }

 
 \section{Conclusion and Outlook}
 
 


\begin{figure}[ht]
	\centering
	\includegraphics[width=0.45\linewidth]{fe_diff_simplefit.pdf}
	\includegraphics[width=0.45\linewidth]{spectrum_59keV.png}
  	\caption{Left: Spectrum of energy of the clusters reconstructed as produced by interactions of 5.9~keV photons in gas. Right: Spectrum of energy of the clusters reconstructed as produced by interactions of 59~keV photons in gas.}
  	\label{fig:55Fe&59keV}
\end{figure}

\begin{figure}[ht]
	\centering
	\includegraphics[width=0.45\linewidth]{energy_spectrum.png}
	\includegraphics[width=0.45\linewidth]{density_spectrum.png}
  	\caption{Spectra of energy (left) and energy density (right) of the clusters reconstructed in three different run types after the preliminary cuts.}
  	\label{fig:ene&dens}
\end{figure}

\begin{figure}[ht]
	\centering
	\includegraphics[width=0.50\linewidth]{density_roc.pdf}
  	\caption{Electron Recoil (ER) signal rejection as a function of the Nuclear Recoil signal detection efficiency.}
  	\label{fig:roc}
\end{figure}
\textcolor{red}{per un paio di punti scriverei vicino il taglio density > ...}

\begin{figure}[ht]
	\centering
	\includegraphics[width=0.45\linewidth]{effS.png}	\includegraphics[width=0.45\linewidth]{effB.png}
  	\caption{Detection efficiency of Nuclear Recoil (NR) signals (left) and Electron Recoil (ER) signals (right) as a function of their reconstructed energy.}
  	\label{fig:effB}
\end{figure}


\begin{figure}[ht]
	\centering
	\includegraphics[width=0.45\linewidth]{energyExt.png}
	\includegraphics[width=0.45\linewidth]{energyExt_cut.png}
  	\caption{Spectra of energy of the clusters reconstructed in three different run types before (left) and after (right) cuts on energy densities.}
  	\label{fig:energy}
\end{figure}
\textcolor{red}{vogliamo provare a dimizzera la bin widt per gli spettri in energia?}
\begin{figure}[ht]
	\centering
	\includegraphics[width=0.50\linewidth]{granchio.png}
  	\caption{Example of a signal reconstructed as a Nuclear Recoil (NR) with an energy of 9~keV.}
  	\label{fig:granchio}
\end{figure}

\textcolor{red}{non farei vedere electron recoil che non ha senso in funzione dell'energia con questo campione }


\bibliography{mybiblio}{}
\bibliographystyle{plain}

\end{document}

