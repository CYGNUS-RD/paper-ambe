The main variable to distinguish signals from and nuclear recoils, and
different types of background, is the energy density of the cluster.
To enhance the purity, a preselection is applied, prior to the tight
selection on $\delta$. Clusters with $l_p>500$ pixels, or $\xi<0.3$
are rejected, to suppress the contribution from cosmic rays. A loose
selection on $\delta>5$ photons/pixel is applied, also to remove the
residual cosmic ray background based on their low specific ionization.
With this selection, the distribution in the 2D plane $\delta$--$l_p$
is shown in Fig.~\ref{fig:dvsl} for data with \ambe source, no source,
and the resulting background-subtracted \ambe data. The latter
distribution shows a clear component of clusters with short length
($l_p\lesssim1$\unit{cm}) and high density ($\delta\gtrsim 10$),
expected from nuclear recoils deposits. In addition, it shows a
smaller component, also present only in the data with \ambe source, of
clusters with a moderate track length, $1.5 \lesssim l_p \lesssim
3.0$\unit{cm}, and a lower energy density than the one characteristic
of the nuclear recoils: $9\lesssim\delta\lesssim12$. Since the density
is inversely proportional to the number of active pixels $n_p$, which
is correlated to the track length, the almost linear decrease of
$\delta$ as a function of $l_p$ points to a component with fixed
energy. The $^{241}$Am is expected to produce photons with
$E=59\keV$. This hypothesis is verified by introducing an oblique
selection in the $\delta-l_p$ plane: $\abs{\delta-y}<2$, where
$y=14-p_l/50$, for the clusters with $120<l_p<250$\unit{pixels},
defining the control region $PR$. The obtained energy spectrum for
these clusters is shown in Fig.~\ref{fig:59keV}, which indeed shows a
peak at $E=65.6\pm3.2\keV$ for these clusters, within the expected
resolution. These events are thus rejected from the nuclear recoils
candidates by vetoing the $PR$ phase space.

\begin{figure}[ht]
  \begin{center}
  \includegraphics[width=0.90\linewidth]{figures/densityvslength_zoom}

  \caption{Supercluster light density $\delta$ versus length $l_p$,
    for data with \ambe source (left), data without any source
    (middle), and the resulting background-subtracted \ambe data.  The
    normalization of data without source is to the same exposure time
    of the \ambe one, accounting for the trigger scale factor
    $\varepsilon_{SF}$, as defined in the text. \label{fig:dvsl}}

  \end{center}
\end{figure}

\begin{figure}[ht]
  \begin{center}
  \includegraphics[width=0.60\linewidth]{figures/calintegral_59keV}

  \caption{Calibrated energy spectrum for candidates in the control
    region $PR$, defined in the text. The background-subtracted
    distribution is fitted with a Gaussian PDF, which shows a mean
    value compatible with $E=59\keV$ expected from the $^{241}$Am
    $\gamma$s interacting with the gas. \label{fig:59keV}}

  \end{center}
\end{figure}

An orthogonal information to the light detected by the sCMOS sensor of
the camera is collected by the PMT, used to trigger the image
shooting. For each image acquired, the corresponding PMT waveform is
recorded.  Tracks from cosmic rays, which typically have a large angle
with respect the cathode plane, as shown in Fig.~\ref{fig:cosmics}
(right), show a broad signal, characterized by the different arrival
times of the clusters along the track produced at different $z$. Conversely,
spot-like signals like \fe deposits or nuclear recoils are characterized by
a short pulse, as shown in Fig.~\ref{fig:waveforms}.
%
\begin{figure}[ht]
  \begin{center}
    \includegraphics[width=0.69\linewidth]{Waveforms.png}

    \caption{Example of two acquired waveforms: one short pulse
  recorded in presence of \fe radioactive source, together with a long
  signal very likely due to cosmic ray
  track.  \label{fig:waveforms}}

  \end{center}
\end{figure}

%
The Time Over Threshold (\textit{TOT}) of the PMT signal is measured,
and shown in Fig.~\ref{fig:pmttot}. It can be seen from the region
around 270\unit{ns}, dominated by the cosmic rays also in the data
with the \ambe source, that the trigger scale factor
$\varepsilon_{SF}$ also holds for the PMT event rate.
%
\begin{figure}[ht]
  \begin{center}
  \includegraphics[width=0.45\linewidth]{figures/pmt_tot}

   \caption{PMT waveform time over threshold ($TOT$).  The last bin
    integrates all the events with $TOT>400$\unit{ns}. Filled points
    represent data with \ambe source, dark gray (light blue)
    distribution represents data with \fe source (no source).  The
    normalization of data without source is to the same exposure time
    of the \ambe one, with trigger scale factor $\varepsilon_{SF}$
    applied. For the data with \fe, a scaling factor of one tenth is
    applied for clearness, given the larger activity of this
    source. \label{fig:pmttot}}

  \end{center}
\end{figure}
%
As expected, spot-like clusters (in 3D) correspond to a short pulse in
the PMT, while cosmic ray tracks have a much larger pulse. The
contribution of cosmic ray tracks is clearly visible in the data with
radioactive sources. A selection on this variable is helpful to
further reject residual comsic rays background present in the \ambe or
\fe data, in particular tracks which may have been split in multiple
superclusters, like the case shown in Fig.~\ref{fig:super_clusters2}
(bottom), and thus passing the above preselection on the cluster
shapes. A selection $TOT<250$\unit{ns} is then added to the event
preselection.  It has an efficiency of 98\% on cluster candidates in
\ambe data (after background subtraction), while it is only 80\%
efficient on data with \fe source.


The light density, and the energy spectrum, after the full
preselection, is shown in Fig.~\ref{fig:presel}.
\begin{figure}[ht]
  \begin{center}
  \includegraphics[width=0.45\linewidth]{figures/density_fullSel}
  \includegraphics[width=0.45\linewidth]{figures/energy_fullSel}

  \caption{Supercluster light density $\delta$ (left) and calibrated
    energy (right), after the full preselection described in the text
    to select nuclear recoil candidates. Filled points represent data
    with \ambe source, dark gray (light blue) distribution represents
    data with \fe source (no source).  The normalization of data
    without source is to the same exposure time of the \ambe one, with
    trigger scale factor $\varepsilon_{SF}$ applied. For the data
    with \fe, a scaling factor of one tenth is applied for clearness,
    given the larger activity of this source.  \label{fig:presel}}

  \end{center}
\end{figure}

The light density, after the preselection, shows a large shape
difference between the data with \ambe source, data with \fe source,
and data without any source.  The cosmics-background-subtracted
distributions of $\delta$ in \ambe data and \fe data, shown in the
bottom panel of Fig.~\ref{fig:presel} (left), are used to evaluate a
curve of electron recoils rejection ($1-\varepsilon^\delta_{B}$) as a
function of signal efficiency ($\varepsilon^\delta_{S}$), obtained
varying the selection on $\delta$. This is shown in
Fig.~\ref{fig:roc}. The same procedure could be applied to estimate
the rejection factor against the cosmic ray induced background, but
this is not shown because of the limited sample without source and
because this kind of background will be negligible when operating the
detector underground. This is shown in Fig.~\ref{fig:roc}. While this
cut-based approach is minimalist, and could be improved by profiting
of the correlations among $\delta$ and the variables used in the
preselection in a multivariate analysis, it shows that a good
rejection factor of electron recoils at $E=6\keV$ can be obtained.
%
\begin{figure}[ht]
  \begin{center}
  \includegraphics[width=0.45\linewidth]{figures/density_roc}

  \caption{Background rejection as a function of the signal
    efficiency, varying the selection on th $\delta$ variable in data
    with either \fe (background sample) or \ambe (signal sample)
    sources.  \label{fig:roc}}

  \end{center}
\end{figure}
%

The preselection efficiency for electron recoils is estimated on
the \fe data sample, and it is measured to be
$\varepsilon_{B}^{presel}=70\%$. Table~\ref{tab:roc} shows then the
full signal efficiency and electrons rejection factor for two example
working points, $\mathrm{WP}_{40}$ and $\mathrm{WP}_{50}$, having 40\%
and 50\% signal efficiency for the selection on $\delta$. They
correspond to a selection $\delta>11$ and $\delta>10$, respectively.


\begin{table*}[t]
\caption{Signal (nuclear recoils) and background (electron recoils) efficiency for
  two different selections on $\delta$.\label{tab:roc}}
\vspace{10pt}
\normalsize
\centering
\begin{tabular}{l c c c | c c c }
  \hline\hline
  working point & \multicolumn{3}{c}{Signal efficiency} & \multicolumn{3}{c}{Background efficiency} \\
  \hline
  & $\varepsilon_{S}^{presel}$ & $\varepsilon_{S}^{\delta}$ & $\varepsilon_{S}^{total}$ & $\varepsilon_{B}^{presel}$ & $\varepsilon_{B}^{\delta}$ & $\varepsilon_{B}^{total}$ \\
  \hline
  $\mathrm{WP}_{50}$  & 0.98                        & 0.51                      & 0.50                     & 0.70                     & 0.050                     & 0.035 \\
  $\mathrm{WP}_{40}$  & 0.98                        & 0.41                      & 0.40                     & 0.70                     & 0.012                     & 0.008 \\
  \hline\hline
\end{tabular}
\end{table*}

As an example, the energy spectrum for the candidates passing the
selection which is 50\% efficient on the \ambe sample is shown in
Fig.~\ref{fig:fullsel_effi} (left). With this selection, the signal
efficiency is computed for both the example working points in bins of
energy. The electron recoil efficiency, $\varepsilon_{B}^{total}$,
represents an overall background efficiency at a fixed energy
$E=6\keV$, characteristic of the \fe emitted photons. For the
$WP_{50}$, the efficiency for very low-energy recoils, $E=6\keV$, is
still 18\%, dropping to almost zero at $E\lesssim4\keV$.

%
\begin{figure}[ht]
  \begin{center}
    \includegraphics[width=0.45\linewidth]{figures/energyFull_WP50}
    \includegraphics[width=0.45\linewidth]{figures/energyFull_effi}

    \caption{Left: supercluster calibrated energy (left), after the
      full selection, which includes $\delta>10$, 50\% efficient on
      signal, to select nuclear recoil candidates. Filled points
      represent data with \ambe source, dark gray (light blue)
      distribution represents data with \fe source (no source).  The
      normalization of data without source is to the same exposure
      time of the \ambe one, with trigger scale factor
      $\varepsilon_{SF}$ applied. For the data with \fe, a scaling
      factor of one tenth is applied for clearness, given the larger
      activity of this source. Right: efficiency for nuclear recoil
      candidates as a function of energy, estimated on \ambe data, for
      two example selections, described in the text, having either 4\%
      or 1\% efficiency on electron recoils at
      $E=6\keV$. \label{fig:fullsel_effi}}

  \end{center}
\end{figure}

Examples of two candidates of nuclear recoils, fulfilling the complete
$\mathrm{WP}_{50}$ selection, with a light density $\delta\gtrsim10$
photons/pixels, and with energies of 5.2 and 6.0\keV, are shown in
Fig.~\ref{fig:lowEnergyNR}. The images are a zoom of the
full-resolution frame, after the pedestal subtraction.

\begin{figure}[ht]
  \begin{center}
  \includegraphics[width=0.49\linewidth]{figures/pic_run02097_ev59_oriIma_paper}
  \includegraphics[width=0.49\linewidth]{figures/pic_run02097_ev317_oriIma_paper}

  \caption{Examples of two nuclear recoil candidates, selected with
    the full selection, shown in a zoom of $100\times100$ pixel
    matrix, after the zero suppression of the image. Left: a candidate
    with $E=5.2\keV$ and $\delta=10.5$, right: a candidate with
    $E=6.0\keV$ and $\delta=10$.  \label{fig:lowEnergyNR}}

\end{center}
\end{figure}

