As mentioned in the previous Section, the 1D observable chosen to
distinguish the signal of nuclear recoils from the various types of
background is the energy density $\delta$ of the cluster.

\subsection{Signal preselection}
To enhance the purity of the signal sample, a preselection was
applied, prior to a tighter selection on $\delta$: clusters with
$l_p>6.3$\unit{cm} or $\xi<0.3$ were rejected to primarily suppress
the contribution from cosmic rays. A further loose requirement
$\delta>5$ photons/pixel was also applied to remove the residual
cosmic rays background based on their low specific ionization. These
thresholds, which only reject very long and narrow clusters, are very
loose for nuclear recoils with $E<1\MeV$ energies, given the expected
range in simulated events, shown in Fig.~\ref{fig:range}, of less than
1\unit{cm}. Thus the preselection efficiency for signal is assumed to
be 100\%. For electron recoils it can be estimated on data by using
the \fe data sample, and is measured to be
$\varepsilon_{B}^{presel}=70\%$. Since the X-ray photo-electrons of
this source are monochromatic, the estimate of the electron recoils
rejection is only checked for an energy around $E=5.9$\keV. The
spectrum of nuclear recoils from \ambe source, instead, extends over a
wider range of energies, around [1--100]\keV.

With this preselection, the distribution in the 2D plane
$\delta$--$l_p$ is shown in Fig.~\ref{fig:dvsl} for \ambe source and
no-source data and for the resulting background-subtracted \ambe data.
The latter distribution shows a clear component of clusters with short
length ($l_p\lesssim1$\unit{cm}) and high density ($\delta\gtrsim
10$), expected from nuclear recoils deposits.

In addition, it shows a smaller component, also present only in the
data with \ambe source, of clusters with a moderate track length,
$1.5 \lesssim l_p \lesssim 3.0$\unit{cm}, and a lower energy density
than the one characteristic of the nuclear recoils
($9\lesssim\delta\lesssim12$). Since the density is inversely
proportional to the number of active pixels $n_p$, which is correlated
to the track length, the almost linear decrease of $\delta$ as a
function of $l_p$ points to a component with fixed energy. The
$^{241}$Am is expected to produce photons with $E=59\keV$. This
hypothesis is verified by introducing an oblique selection in the
$\delta-l_p$ plane: $\abs{\delta-y}<2$, where $y=14-p_l/50$, for the
clusters with $120<l_p<250$\unit{pixels}, defining the \textit{photon
control region}, $PR$. The approximate oblique region in the
$\delta-l_p$ plane corresponding to $PR$ is also shown in
Fig.~\ref{fig:dvsl}.  The obtained energy spectrum for these clusters
is shown in Fig.~\ref{fig:59keV}, which indeed shows a maximum at
$E=60.9\pm3.6\keV$, within the expected resolution. These events are
thus rejected from the nuclear recoils candidates by vetoing the $PR$
phase space.

\begin{figure}[ht]
  \begin{center}
  \includegraphics[width=0.90\linewidth]{figures/densityvslength_zoom}

  \caption{Supercluster light density $\delta$ versus length $l_p$,
    for data with \ambe source (left), data without any artificial
    source (middle), and the resulting background-subtracted \ambe
    data.  The normalization of data without source is to the same
    exposure time of the \ambe one, accounting for the trigger scale
    factor $\varepsilon_{SF}$, as defined in the text. The orange
    ellipse represents the approximate contour of the 59\keV photons
    control region ($PR$) defined in the text. \label{fig:dvsl}}

  \end{center}
\end{figure}

\begin{figure}[ht]
  \begin{center}
  \includegraphics[width=0.60\linewidth]{figures/calintegral_59keV}

  \caption{Calibrated energy spectrum for candidates in the control
    region $PR$, defined in the text. The background-subtracted
    distribution is fitted with a Gaussian PDF, which shows a mean
    value compatible with $E=59\keV$ originated from the $^{241}$Am
    $\gamma$s interaction within the gas. \label{fig:59keV}}

  \end{center}
\end{figure}

\subsection{PMT-based cosmic ray suppression}
An independent information to the light detected by the sCMOS sensor
of the camera is obtained from the PMT pulse, used to trigger the
image shooting. For each image acquired, the corresponding PMT pulse
waveform is recorded.  Tracks from cosmic rays, which typically have a
large angle with respect the cathode plane, as shown in
Fig.~\ref{fig:cosmics} (right), show a broad PMT waveform,
characterized by different arrival times of the several ionization
clusters produced along the track at different $z$. Conversely,
spot-like signals like \fe deposits or nuclear recoils are
characterized by a short pulse, as shown in Fig.~\ref{fig:waveforms}.
%
\begin{figure}[ht]
  \begin{center}
    \includegraphics[width=0.69\linewidth]{figures/Waveforms.png}

    \caption{Example of two acquired waveforms: one short pulse
  recorded in presence of \fe radioactive source, together with a long
  signal very likely due to a cosmic ray
  track.  \label{fig:waveforms}}

  \end{center}
\end{figure}

%
The Time Over Threshold (\textit{TOT}) of the PMT pulse was measured,
and is shown in Fig.~\ref{fig:pmttot}. It can be seen from the region
around 270\unit{ns}, dominated by the cosmic rays also in the data
with the \ambe source, that the trigger scale factor
$\varepsilon_{SF}$ also holds for the PMT event rate.
%
\begin{figure}[ht]
  \begin{center}
  \includegraphics[width=0.45\linewidth]{figures/pmt_tot}

   \caption{PMT waveform time over threshold ($TOT$).  The last bin
    integrates all the events with $TOT>400$\unit{ns}. Filled points
    represent data with \ambe source, dark gray (light blue)
    distribution represents data with \fe source (no source).  The
    normalization of data without source is to the same exposure time
    of the \ambe one, with trigger scale factor $\varepsilon_{SF}$
    applied. For the data with \fe, a scaling factor of one tenth is
    applied for clearness, given the larger activity of this
    source. \label{fig:pmttot}}

  \end{center}
\end{figure}
%
As expected, spot-like clusters (in 3D) correspond to a short pulse in
the PMT, while cosmic ray tracks have a much larger pulse. The
contribution of cosmic ray tracks is clearly visible in the data with
radioactive sources. A selection on this variable is helpful to
further reject residual cosmic rays background present in the \ambe or
\fe data, in particular tracks which may have been split in multiple
superclusters, like the case shown in Fig.~\ref{fig:super_clusters2}
(bottom), and thus passing the above preselection on the cluster
shapes. A selection $TOT<250$\unit{ns} is then imposed.  It has an
efficiency of 98\% on cluster candidates in
\ambe data (after muon-induced background subtraction), while it is only 80\%
efficient on data with \fe source. This larger value is expected
because of the residual contamination of signals from cosmic rays,
which fulfill the selection because their track is split in multiple
sub-clusters, or because they are only partially visible in the sCMOS
sensor image. These can be eventually detected as long, in the time
dimension, by the PMT.  The light density and the energy spectrum of
the preselected clusters are shown in Fig.~\ref{fig:presel}.
%
\begin{figure}[ht]
  \begin{center}
  \includegraphics[width=0.45\linewidth]{figures/density_fullSel}
  \includegraphics[width=0.45\linewidth]{figures/energy_fullSel}

  \caption{Supercluster light density $\delta$ (left) and calibrated
    energy $E$ (right), after the preselection and cosmic ray
    suppression described in the text to select nuclear recoil
    candidates. Filled points represent data with \ambe source, dark
    gray (light blue) distribution represents data with \fe source
    (no-source).  The normalization of no-source data is to the same
    exposure time of the \ambe data, with the trigger scale factor
    $\varepsilon_{SF}$ applied. For the data with \fe, a scaling
    factor of one tenth is applied for clearness, given the larger
    activity of this source.  \label{fig:presel}}

  \end{center}
\end{figure}


\subsection{Light density and  \fe events rejection}
The light density distribution, after the above preselection and
cosmic ray suppression, appear to be different among the data
with \ambe source, data with \fe source, and data without any
artificial source.  The cosmic-background-subtracted distributions of
$\delta$ in \ambe data and \fe data, shown in the bottom panel of
Fig.~\ref{fig:presel} (left), are used to evaluate a curve of 5.9\keV
electron recoils rejection ($1-\varepsilon^\delta_{B}$) as a function
of signal efficiency ($\varepsilon^\delta_{S}$), obtained varying the
selection on $\delta$, shown in Fig.~\ref{fig:roc}.  The same
procedure could be applied to estimate the rejection factor against
the cosmic ray induced background, but this is not shown because of
the limited size of the no-source data. This kind of background will
however be negligible when operating the detector underground, in the
context of the \cygno project, so no further estimates are given for
this source.

Table~\ref{tab:roc} shows the full signal efficiency and electrons
rejection factor for two example working points, $\mathrm{WP}_{40}$
and $\mathrm{WP}_{50}$, having 40\% and 50\% signal efficiency for the
selection on $\delta$, averaged over the full energy spectrum
exploited in the \ambe data. They correspond to a selection
$\delta>11$ and $\delta>10$, respectively.  While this cut-based
approach is minimalist, and could be improved by profiting of the
correlations among $\delta$ and the observables used in the
preselection in a more sophisticated multivariate analysis, it shows
that a rejection factor approximately in the range
[$10^{-3}$-$10^{-2}$] of electron recoils at $E=5.9\keV$ with a
gaseous detector at atmospheric pressure can be obtained, while
retaining a high fraction of signal events.
%
\begin{figure}[ht]
  \begin{center}
  \includegraphics[width=0.45\linewidth]{figures/density_roc}

  \caption{Background rejection as a function of the signal
    efficiency, varying the selection on the $\delta$ variable in data
    with either \fe (background sample) or \ambe (signal sample)
    sources.  \label{fig:roc}}

  \end{center}
\end{figure}
%



\begin{table*}[t]

\caption{Signal (nuclear recoils induced by \ambe radioactive source) and background (photo-electron recoils of X-rays
         with $E=5.9$\keV from \fe radioactive source) efficiency for
         two different selections on $\delta$.\label{tab:roc}}

\vspace{10pt}
\normalsize
\centering
\begin{tabular}{l c c c | c c c }
  \hline\hline
  working point & \multicolumn{3}{c}{Signal efficiency} & \multicolumn{3}{c}{Background efficiency} \\
  \hline
  & $\varepsilon_{S}^{presel}$ & $\varepsilon_{S}^{\delta}$ & $\varepsilon_{S}^{total}$ & $\varepsilon_{B}^{presel}$ & $\varepsilon_{B}^{\delta}$ & $\varepsilon_{B}^{total}$ \\
  \hline
  $\mathrm{WP}_{50}$  & 0.98                        & 0.51                      & 0.50                     & 0.70                     & 0.050                     & 0.035 \\
  $\mathrm{WP}_{40}$  & 0.98                        & 0.41                      & 0.40                     & 0.70                     & 0.012                     & 0.008 \\
  \hline\hline
\end{tabular}
\end{table*}



\subsection{Nuclear recoils energy spectrum and differential efficiency}

The energy spectrum for the candidates with
$\varepsilon_{S}^{total}$=50\% in the \ambe sample is shown in
Fig.~\ref{fig:fullsel_effi} (left).  The signal efficiency is then
computed for both the example working points in bins of the visible
energy. The efficiency, $\varepsilon_{B}^{total}$, represents a
$\gamma$ background efficiency at a fixed energy $E=5.9\keV$, \ie, the
energy of the photons emitted by the \fe source. For the
$\mathrm{WP}_{50}$, the efficiency for very low-energy recoils,
$E=5.9\keV$, is still 18\%, dropping to almost zero at
$E\lesssim4\keV$.

\begin{figure}[ht]
  \begin{center}
    \includegraphics[width=0.45\linewidth]{figures/energyFull_WP50}
    \includegraphics[width=0.45\linewidth]{figures/energyFull_effi}

    \caption{Left: supercluster calibrated energy $E$ (left), after the
      full selection, which includes $\delta>10$, 50\% efficient on
      signal, to select nuclear recoil candidates. Filled points
      represent data with \ambe source, dark gray (light blue)
      distribution represents   \fe source (no-source) data.  The
      normalization of no-source  data  is to the same exposure
      time of the \ambe data, with the trigger scale factor
      $\varepsilon_{SF}$ applied. For the  \fe data, a scaling
      factor of one tenth is applied for clearness, given the larger
      activity of this source. Right: efficiency for nuclear recoil
      candidates as a function of energy, estimated on \ambe data, for
      two example selections, described in the text, having either 4\%
      or 1\% efficiency on electron recoils at
      $E=5.9\keV$. \label{fig:fullsel_effi}}

  \end{center}
\end{figure}

Two candidate nuclear recoils images, fulfilling the
$\mathrm{WP}_{50}$ selection (with a light density $\delta\gtrsim10$
photons/pixels and with energies of 5.2 and 6.0\keV) are shown in
Fig.~\ref{fig:lowEnergyNR}. The displayed images are a portion of the
full-resolution frame, after the pedestal subtraction. While the
determination of the direction of detected nuclear recoil is still
under study, it appears pretty clear from the image that some
sensitivity to their direction, even at such low energies, is retained
and can be further exploited.

\begin{figure}[ht]
  \begin{center}
  \includegraphics[width=0.49\linewidth]{figures/pic_run02097_ev59_oriIma_paper}
  \includegraphics[width=0.49\linewidth]{figures/pic_run02097_ev317_oriIma_paper}

  \caption{Examples of two nuclear recoil candidates, selected with  the full selection, shown in a portion of $100\times100$ pixel matrix, after the zero suppression of the image. Left: a candidate
    with $E=5.2\keV$ and $\delta=10.5$, right: a candidate with
    $E=6.0\keV$ and $\delta=10$.  \label{fig:lowEnergyNR}}

\end{center}
\end{figure}

