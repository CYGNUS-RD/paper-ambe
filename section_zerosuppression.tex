The electronic noise of the sensor was estimated in data-taking runs acquired with
the sensor in complete dark ({\it pedestal} runs). For each pixel, the
pedestal was computed as the average of the counts over many frames,
while the electronic noise was estimated as their standard deviation
(SD). The distribution of the pixels SD is shown in
Fig.~\ref{fig:noise}. The mode of this distribution is about 1.8
photons per pixels, but a tail is present, with pixels having a noise
of more than 5 photons per pixels. For such pixels, a very
non-Gaussian distribution was observed, while for the pixels in the
bulk of the distribution, the pedestal distribution followed a Gaussian
shape. To form the pedestal-subtracted image, the pedestal mean
$\mu_i$ was subtracted to the image for each $i^{th}$ pixel.  An initial
noise suppression was applied by neglecting the pixels with counts less
than $1.3\times\textrm{SD}_i$.
%where $i$ represents the
%pixel (\textit{zero suppression}). 
%
\begin{figure}[ht]
  \centering
  \includegraphics[width=0.45\linewidth]{figures/sensor_noise}
  \caption{Distribution of the electronic noise of the sensor,
    estimated in images taken with sensor in complete dark, and
    evaluated as the SD of the distribution of the counts for each
    pixel.  \label{fig:noise}}
\end{figure}
%
On such pedestal-subtracted zero-suppressed images an upper threshold
was applied to reject hot pixels, which are more likely due to sensor
instabilities than to energy deposition. These were found  to be not
malfunctioning pixels since they disappeared after a power cycle of the
camera: therefore a dynamic (run-by-run) suppression was needed.
They were efficiently identified as high-intensity, isolated pixels,
and distinguished by a true energy deposit, for which each pixel is
surrounded by some other active pixels. A threshold was applied on the
ratio $R_9$ between the pixel and the average of the counts in a
$3\times3$ pixels matrix surrounding it, and a minimum number of two
pixels above noise in that matrix was required to discriminate  good from
hot pixels.

The resolution of the resulting image was then reduced by forming
\textit{macro-pixels}, by averaging the counts in $4\times4$ pixel
matrices. This was needed to reduce the combinatorics   of the subsequent
clustering algorithm to be executed   in a reasonable time for each
image. On such $512\times512$ pixel map, a median
filtering~\cite{medianfilter} was applied, as described in more details
in Ref.~\cite{medianfilter_cygno}. The output image is passed to the
basic clustering algorithm, described in the following.


