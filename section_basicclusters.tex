The basic clustering algorithm, called \idbscan and  described in 
details in Ref.~\cite{iDBSCAN}, represents an evolution of the
neighboring pixels clusters, called \nnc, previously used to study the
performances of the \lemon detector with \fe radioactive
source~\cite{bib:fe55}. It is briefly described also here, since it
represents the seeding for the final clustering algorithm.

The energy deposition in the sensitive volume of the TPC was estimated
from the two-dimensional (2D) projection on the $x$--$y$ axes of the
light emitted in the multiplication process within the GEMs
planes. The pattern showed a large variation, depending on the interacting
particle. For events of \fe calibration source, the signature of the
typical 5.9\keV photons was a spot of few mm$^2$ with the exact size
depending on the diffusion in the gas, \ie, on the distance from the
anode along $z$ of the energy deposition (see Fig.~\ref{fig:signals}
left). Cosmic rays travel across the volume and leave a typical
signature of a straight track, shown in Fig.~\ref{fig:typicalimage1}
(right), but with several agglomerations with larger density along the
path. Finally, natural radioactivity and the signal from nuclear
recoils due to neutrons originated by the \ambe source showed an
irregular pattern, sometimes curly, with several kinks along the
path. Their track length and their size was found to depend  a lot on the initial
energy of the impinging neutron, and also on the mass of the recoiling
nucleus.

Thus, the clustering algorithm needs to be flexible enough to
efficiently reconstruct a diverse set of patterns, from small round
spots to long and kinky tracks. A first step of the clustering,
called \textit{seeding}, was used: it focused in the clustering of spot-like
neighboring pixels.  The method applied for the \lemon detector is an
evolution of the classic \dbscan algorithm~\cite{dbscan}.  This is a
non-parametric, density-based clustering, which groups together pixels
above threshold with many neighbors. Its distinctive characteristics
making this method very suitable to the \lemon case is its ability to
label as outliers, and so not to include in the clusters, pixels that
lie isolated in low-density regions, \ie, pixels from electronic noise
of the sensor surviving the zero suppression. The extension of \dbscan
used for \lemon data analysis consists in a larger phase space for the
points that includes not only the $x$--$y$ plane, but also the number
of photons in each pixel $N_{ph}$ (\ie, the light intensity measured in
each pixel).

To be as inclusive as possible, and since different interactions may
have vastly different intensities, even varying along the track, the
clustering procedure was iterated three times.  First, the \dbscan
parameters were tuned to form clusters of dense (in $x$--$y$ dimension)
and intense (in the $N_{ph}$ dimension) pixels. The density in 3D was
called \textit{sparsity}.  This step typically identifies either rare
hot spots of the GEMs, or, efficiently, short nuclear recoils. The
pixels belonging to the reconstructed clusters were then removed from
the image, and the \dbscan procedure was repeated, with looser sparsity
parameters. The second iteration was tuned to efficiently reconstruct
\fe round spots and slices of tracks from nuclear recoils with lower
intensity. It also collected the agglomerations with larger density
along cosmic tracks, clearly visible in the example in
Fig.~\ref{fig:typicalimage1} (right).  A third iteration of \dbscan
with even looser parameters was finally executed, targeting faint
portions of a cluster. These were especially used as a proxy for
the  characterization of clustered noisy pixels.

To be computationally viable, the \idbscan basic clustering was
performed on the image with reduced resolution, 512$\times$512. In
typical images this allows the basic clusters reconstruction to be run
in approximately 1\unit{s} on an \textit{Intel Xeon E5-2620
2.00\unit{GHz}} and 64\unit{GB} RAM. The reconstruction algorithm is
implemented in \PYTHONthree~\cite{python3}, and interfaced with the
CERN \ROOT v.6~\cite{root}.


Examples of clustered pixels in two cases are shown in
Fig.~\ref{fig:basic_clusters}. The left panel shows an example of
clusters reconstructed on the low-resolution image of one event
with \fe source. Three spots are clearly visible: one, as typical for
events with this calibration source with a moderate activity, is
reconstructed by a single cluster of the second iteration. The other
two are close enough that are merged in a single cluster of the same
iteration. The right panel shows the outcome of the \idbscan algorithm
on a longer track presumably from natural radioactivity and one
possible short nuclear recoil.  The nuclear recoil candidate is very
dense, high-energetic, and isolated, and it is reconstructed as a single
cluster in the first iteration. The long track shows several clusters
with higher intensity. One of them has a large energy, and it is
reconstructed as an isolated single iteration-1 cluster. The rest of
the track is reconstructed by multiple iteration-2 clusters, which are
split where the energy deposition has a minimum for too many pixels to
be joined together in the same clusters. Events like these, which are
frequent for cosmic rays, natural radioactivity, but also signals from
nuclear recoils with higher energy, justify the need of the subsequent
step of the \textit{superclustering}, which follows the track pattern
without splitting it in parts. This is described in the following
section.
%
\begin{figure}[ht]
  \begin{center}
     \includegraphics[width=0.49\linewidth]{figures/pic_run01843_ev93_2nd_3D_paper}
      \includegraphics[width=0.49\linewidth]{figures/pic_run02317_ev8_all_3D_paper}
      \caption{Basic clusters reconstructed with the \idbscan
    algorithm in the low resolution (512$\times$512) image for two
    example events with very different patterns. Left: clusters on
    spots from \fe source, two of which are merged together. Right:
    Track from natural radioactivity and a nuclear recoil candidate in
    an event with \ambe source. The long track is split in several
    basic clusters of different \idbscan
    iteration. \label{fig:basic_clusters}}
  \end{center}
\end{figure}
